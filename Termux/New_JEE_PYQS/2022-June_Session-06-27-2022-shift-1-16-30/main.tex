%2022-June Session-06-27-2022-shift-1-16-30
\let\negmedspace\undefined
\let\negthickspace\undefined
\documentclass[journal]{IEEEtran}
\usepackage[a5paper, margin=10mm, onecolumn]{geometry}
%\usepackage{lmodern} % Ensure lmodern is loaded for pdflatex
\usepackage{tfrupee} % Include tfrupee package

\setlength{\headheight}{1cm} % Set the height of the header box
\setlength{\headsep}{0mm}     % Set the distance between the header box and the top of the text

\usepackage{gvv-book}
\usepackage{gvv}
\usepackage{cite}
\usepackage{amsmath,amssymb,amsfonts,amsthm}
\usepackage{algorithmic}
\usepackage{graphicx}
\usepackage{textcomp}
\usepackage{xcolor}
\usepackage{txfonts}
\usepackage{listings}
\usepackage{enumitem}
\usepackage{mathtools}
\usepackage{gensymb}
\usepackage{comment}
\usepackage[breaklinks=true]{hyperref}
\usepackage{tkz-euclide} 
\usepackage{listings}
% \usepackage{gvv}                                        
\def\inputGnumericTable{}                                 
\usepackage[latin1]{inputenc}                                
\usepackage{color}                                            
\usepackage{array}                                            
\usepackage{longtable}                                       
\usepackage{calc}                                             
\usepackage{multirow}                                         
\usepackage{hhline}                                           
\usepackage{ifthen}                                           
\usepackage{lscape}

\renewcommand{\thefigure}{\theenumi}
\renewcommand{\thetable}{\theenumi}
\setlength{\intextsep}{10pt} % Space between text and floats


\numberwithin{equation}{enumi}
\numberwithin{figure}{enumi}
\renewcommand{\thetable}{\theenumi}

% Marks the beginning of the document
\begin{document}
\bibliographystyle{IEEEtran}

\title{Assignment 2 \\ 2022-June Session-06-27-2022-shift-1:16-30}
\author{EE24BTECH11049 \\ Patnam Shariq Faraz Muhammed}

% \maketitle
% \newpage
% \bigskip
{\let\newpage\relax\maketitle}
\section*{MCQ}
\begin{enumerate}
    
    %1st Question 
    \item 
    Five numbers $x_1, x_2, x_3, x_4, x_5$ are randomly selected from the numbers $1, 2, 3,\dots, 18$ and are arranged in the increasing order $\brak{x_1 < x_2 < x_3 < x_4 < x_5}$. The probability that $x_2 = 7$ and $x_4 = 11$ is:

    \hfill{\brak{\text{2022-Jun}}}
    
    \begin{enumerate}
    \begin{multicols}{4}
        \item $\frac{1}{136}$
        \item $\frac{1}{72}$
        \item $\frac{1}{68}$
        \item $\frac{1}{34}$
    \end{multicols}
    \end{enumerate}

    %2nd Question
    \item 
    Let $X$ be a random variable having binomial distribution $\vec{B}\brak{7, p}$. If $\vec{P}\brak{X = 3} = 5\vec{P}\brak{X = 4}$, then the sum of the mean and the variance of $X$ is: 

    \hfill{\brak{\text{2022-Jun}}}
    
    \begin{enumerate}
    \begin{multicols}{4}
        \item $\frac{105}{16}$
        \item $\frac{7}{16}$
        \item $\frac{77}{36}$
        \item $\frac{49}{16}$
    \end{multicols}
    \end{enumerate}

    %3rd Question
    \item 
    The value of 
    \begin{align*}
        \cos{\brak{\frac{2\pi}{7}}} + \cos{\brak{\frac{4\pi}{7}}} + \cos{\brak{\frac{6\pi}{7}}} 
    \end{align*}
    is equal to;

    \hfill{\brak{\text{2022-Jun}}}
    
    \begin{enumerate}
    \begin{multicols}{4}
        \item $-1$
        \item $-\frac{1}{2}$
        \item $-\frac{1}{3}$
        \item $-\frac{1}{4}$
    \end{multicols}
    \end{enumerate}

    %4th Question
    \item 
    \begin{align*}
        \sin^{-1}{\brak{\sin{\frac{2\pi}{3}}}} + \cos^{-1}{\brak{\cos{\frac{7\pi}{6}}}} + \tan^{-1}{\brak{\tan{\frac{3\pi}{4}}}}
    \end{align*}
    is equal to;

    \hfill{\brak{\text{2022-Jun}}}
    
    \begin{enumerate}
    \begin{multicols}{4}
        \item $\frac{11\pi}{12}$
        \item $\frac{17\pi}{12}$
        \item $\frac{31\pi}{12}$
        \item $-\frac{3\pi}{4}$
    \end{multicols}
    \end{enumerate}

    %5th Question
    \item 
    The boolean expression $\brak{ \sim \brak{p \land q}} \lor q$ is equivalent to: 

    \hfill{\brak{\text{2022-Jun}}}
    
    \begin{enumerate}
    \begin{multicols}{2}
        \item $q \to \brak{p \land q}$
        \item $p \to q$
        \item $p \to \brak{p \to q}$
        \item $p \to \brak{p \lor q}$
    \end{multicols}
    \end{enumerate}

\end{enumerate}

\section*{INTEGER}

\begin{enumerate}

    %1st Question
    \item 
    let $f: \mathbf{R} \mapsto \mathbf{R}$ be a function defined by 
    \begin{align*}
        f\brak{x} = \frac{2e^{2x}}{e^{2x} + e^{x}}
        \text{ then }
        f\brak{\frac{1}{100}} + f\brak{\frac{2}{100}} + f\brak{\frac{3}{100}} + \dots + f\brak{\frac{99}{100}}
    \end{align*}
    is equal to \rule{1cm}{0.1pt}

    \hfill{\brak{\text{2022-Jun}}}

    %2nd Question 
    \item
    If the sum of all the roots of the equation 
    \begin{align*}
        e^{2x} - 11e^x - 45e^{-x} +\frac{81}{2}=0
    \end{align*}
    is $\log_e p$, then $p$ is equal to \rule{1cm}{0.1pt}

    \hfill{\brak{\text{2022-Jun}}}

    %3rd Question 
    \item 
    The positive value of the determinant of the matrix $A$, whose
    \begin{align*}
        \text{adj}\brak{\text{adj}\brak{A}} = \myvec{14 & 28 & -14 \\ -14 & 14 & 28 \\ 28 & -14 & 14},
    \end{align*}
    is \rule{1cm}{0.1pt}

    \hfill{\brak{\text{2022-Jun}}}

    %4th Question 
    \item 
    The number of ways, $16$ identical cubes, of which $11$ are blue and rest are red, can be placed in a row so that between any two red cubes there should be at least $2$ blue cubes, is \rule{1cm}{0.1pt}

    \hfill{\brak{\text{2022-Jun}}}

    %5th Question 
    \item 
    If the coefficient of $x^{10}$ in the binomial expansion of 
    \begin{align*}
        \brak{\frac{\sqrt{x}}{5^{\frac{1}{4}}} + \frac{\sqrt{5}}{x^{\frac{1}{3}}}}^{60}
    \end{align*}
    is $5^kl$ where $l,k \in \mathbf{N}$ and $l$ is co-prime to 5, then $k$ is equal to \rule{1cm}{0.1pt}

    \hfill{\brak{\text{2022-Jun}}}

    %6th Question 
    \item 
    \begin{align*}
        A_1 = \cbrak{\brak{x,y} : \abs{x} \leq y^2, \abs{x} + 2y \leq 8} 
        \text{ and }
        A_2 = \cbrak{\brak{x,y} : \abs{x} + \abs{y} \leq k}.
    \end{align*}
    if $27\text{Area}\brak{A_1} = 5\text{Area}\brak{A_2}$, then $k$ is equal to: \rule{1cm}{0.1pt}

    \hfill{\brak{\text{2022-Jun}}}

    %7th Question 
    \item 
    If the sum of the first ten terms of the series 
    \begin{align*}
        \frac{1}{5} + \frac{2}{65} + \frac{3}{325} + \frac{4}{1025} + \frac{5}{2501} + \dots \text{ is } \frac{m}{n},
    \end{align*}
    where $m$ and $n$ are co-prime numbers, then $m + n$ is equal to \rule{1cm}{0.1pt}

    \hfill{\brak{\text{2022-Jun}}}

    %8th Question 
    \item 
    A rectangle $R$ with end points of one of its sides as \brak{1, 2} and \brak{3, 6} s inscribed in a circle. If the equation of a diameter of the circle is  $2x - y + 4 = 0$ , then the area of $R$ is \rule{1cm}{0.1pt}

    \hfill{\brak{\text{2022-Jun}}}

    %9th Question 
    \item 
    A circle of radius $2$ unit passes through the vertex and the focus of the parabola $y^2 =  2x$ and touches the parabola $y = \brak{x-\frac{1}{4}}^2 + \alpha$ where $\alpha > 0$. Then $\brak{4\alpha - 8}^2$ is equal to \rule{1cm}{0.1pt}

    \hfill{\brak{\text{2022-Jun}}}

    %10th Question 
    \item 
    Let the mirror image of the point $\brak{a, b, c}$ with respect to the plane $ 3x - 4y + 12z + 19 = 0 $ be $\brak{\alpha - 6, \beta, \gamma}$. if $a + b + c = 5$, then $7\beta - 9\gamma$ is equal to \rule{1cm}{0.1pt}
    
    \hfill{\brak{\text{2022-Jun}}}
    
\end{enumerate}
\end{document}
