\let\negmedspace\undefined
\let\negthickspace\undefined
\documentclass[journal]{IEEEtran}
\usepackage[a5paper, margin=10mm, onecolumn]{geometry}
%\usepackage{lmodern} % Ensure lmodern is loaded for pdflatex
\usepackage{tfrupee} % Include tfrupee package

\setlength{\headheight}{1cm} % Set the height of the header box
\setlength{\headsep}{0mm}  % Set the distance between the header box and the top of the text

\usepackage{gvv-book}
\usepackage{gvv}
\usepackage{cite}
\usepackage{amsmath,amssymb,amsfonts,amsthm}
\usepackage{algorithmic}
\usepackage{graphicx}
\usepackage{textcomp}
\usepackage{xcolor}
\usepackage{txfonts}
\usepackage{listings}
\usepackage{enumitem}
\usepackage{mathtools}
\usepackage{gensymb}
\usepackage{comment}
\usepackage[breaklinks=true]{hyperref}
\usepackage{tkz-euclide} 
\usepackage{listings}
% \usepackage{gvv}                                        
\def\inputGnumericTable{}                                 
\usepackage[latin1]{inputenc}                                
\usepackage{color}                                            
\usepackage{array}                                            
\usepackage{longtable}                                       
\usepackage{calc}                                             
\usepackage{multirow}                                         
\usepackage{hhline}                                           
\usepackage{ifthen}                                           
\usepackage{lscape}
\begin{document}

\bibliographystyle{IEEEtran}
\vspace{3cm}

\title{Assignment-7}
\author{EE24BTECH11049}

% \maketitle
% \newpage
% \bigskip
{\let\newpage\relax\maketitle}

\renewcommand{\thefigure}{\theenumi}
\renewcommand{\thetable}{\theenumi}
\setlength{\intextsep}{10pt} % Space between text and floats

\begin{enumerate}
\setcounter{enumi}{15}
    \item Let $\frac{x^2}{a^2}+\frac{y^2}{b^2}=1,a\textgreater b$ be an ellipse, whose eccentricity is $\frac{1}{\sqrt{2}}$ and the length of latus rectum is $\sqrt{14}$. Then the square of the eccentricity of $\frac{x^2}{a^2}-\frac{y^2}{b^2}=1$ is $\colon$
    \hfill{[Feb-2024]}
        \begin{enumerate}
            \item $3$
            \item $\frac{7}{2}$
            \item $\frac{3}{2}$
            \item $\frac{5}{2}$
        \end{enumerate}
    \item Let $3,a,b,c$ be in $A.P.$ and $3,a-1,b+1,c+9$ be in $G.P.$ Then, the arithmetic mean of $a,b$ and $c$ is $\colon$
    \hfill{[Feb-2024]}
        \begin{enumerate}
            \item $-4$
            \item $-1$
            \item $13$
            \item $11$
        \end{enumerate}
    \item Let $C\colon x^2+y^2=4$ and $C\prime\colon x^2+y^2-4\lambda x+9=0$ be two circles.If the set of all values of $\lambda$ so that the circles $C$ and $C\prime$ intersect at two distinct points, is $R-\sbrak{a,b}$, then the point $\brak{8a+12,16b-20}$ lies on the curve $\colon$
    \hfill{[Feb-2024]}
        \begin{enumerate}
            \item $x^2+2y^2-5x+6y=3$
            \item $5x^2-y=-11$
            \item $x^2-4y^2=7$
            \item $6x^2+y^2=42$
        \end{enumerate}
    \item Let $5f\brak{x}+4f\brak{\frac{1}{x}}=x^2-2,\forall x\neq 0$ and $y=9x^2f\brak{x}$, then $y$ is strictly increasing in $\colon$
    \hfill{[Feb-2024]}
        \begin{enumerate}
            \item $\brak{0,\frac{1}{\sqrt{5}}}\cup \brak{\frac{1}{\sqrt{5}},\infty}$
            \item $\brak{-\frac{1}{\sqrt{5}},0}\cup \brak{\frac{1}{\sqrt{5}},\infty}$
            \item $\brak{-\frac{1}{\sqrt{5}},0}\cup \brak{0,\frac{1}{\sqrt{5}}}$
            \item $\brak{-\infty,-\frac{1}{\sqrt{5}}}\cup \brak{0,\frac{1}{\sqrt{5}}}$
        \end{enumerate}
    \item If the shortest distance between the lines $\frac{x-\lambda}{-2}=\frac{y-2}{1}=\frac{z-1}{1}$ and $\frac{x-\sqrt{3}}{1}=\frac{y-1}{-2}=\frac{z-2}{1}$ is $1$, then the sum of all possible values of $\lambda$ is $\colon$ 
    \hfill{[Feb-2024]}
        \begin{enumerate}
            \item $0$
            \item $2\sqrt{3}$
            \item $3\sqrt{3}$
            \item $-2\sqrt{3}$
        \end{enumerate}
    \item If $x=x\brak{t}$ is the solution of the differential eqution\\
    $\brak{t+1}dx=\brak{2x+\brak{t+1}^4}dt,x\brak{0}=2,$ then, $x\brak{1}$ equals \dots
    \hfill{[Feb-2024]}
    \item The number of elements in the set\\
            $S=\cbrak{\brak{x,y,z}\colon x,y,z\in Z,x+2y+3z=42,x,y,z\geq0}$ equals \dots
            \hfill{[Feb-2024]}
    \item If the coefficient of $x^{30}$ in the expansion of $\brak{1+\frac{1}{x}}^6\brak{1+x^2}^7\brak{1-x^3}^8;x\neq0$ is $\alpha$ then $\abs{\alpha}$ equals \dots
    \hfill{[Feb-2024]}
    \item Let $3,11,15,\dots,403$ and $2,5,8,11,\dots,404$ be two arithmetic progressions. Then the sum, of the common terms in them,is equal to \dots
    \hfill{[Feb-2024]}
    \item Let $\cbrak{x}$ denote the fractional part of $x$ and $f\brak{x}=\frac{\cos^{-1}\brak{1-\cbrak{x}^2}\sin^{-1}\brak{1-\cbrak{x}}}{\cbrak{x}-\cbrak{x}^3},x\neq0$. If $L$ and $R$ respectively denote the left and right hand limit of $f\brak{x}$ at $x=0$, then $\frac{32}{\pi^2}\brak{L^2+R^2}$ is equal to \dots
    \hfill{[Feb-2024]}
    \item Let the line $L\colon\sqrt{2}x+y=\alpha$ passes through the point of the intersection $P\brak{in\,the\,first\,quadrant}$ of the circle $x^2+y^2=3$ and the parabola $x^2=2y$. Let the line $L$ touch two circles $C_1$ and $C_2$ of equal radius $2\sqrt{3}$. If the centres $Q_1$ and $Q_2$ of the circles $C_1$ and $C_2$ lie on the $y-axis$, then the square of the area of triangle $PQ_1Q_2$ is equal to \dots 
    \hfill{[Feb-2024]}
    \item Let $P=\cbrak{z\in C\colon\abs{z+2-3i}\leq 1}$ and $Q=\cbrak{z\in C\colon\abs{z\brak{1+i}+\Bar{z}\brak{1-i}}\leq -8}$. Let in $P\cap Q,\abs{z-3+2i}$ be maximum and minimum at $z_1$ and $z_2$ respectively. If $\abs{z_1}^2+2\abs{z}^2=\alpha+\beta\sqrt{2}$, where $\alpha,\beta$ are integers, then $\alpha+\beta$ equals \dots
    \hfill{[Feb-2024]}
    \item If $\int_{-\frac{\pi}{2}}^{\frac{\pi}{2}}\frac{8\sqrt{2}\cos{x}}{\brak{1+e^{\sin{x}}}\brak{1+\sin^4{x}}} \, dx=\alpha\pi+\beta\log_{e}\brak{3+2\sqrt{2}}$, where $\alpha,\beta$ are integers, then $\alpha^2+\beta^2$ equals \dots
    \hfill{[Feb-2024]}
    \item Let the line of shortest distance between the lines\\
            $L_1\colon\overrightarrow{r}=\brak{\hat{i}+2\hat{j}+3\hat{k}}+\lambda\brak{\hat{i}-\hat{j}+\hat{k}}$ and \\
            $L_2\colon\overrightarrow{r}=\brak{4\hat{i}+5\hat{j}+6\hat{k}}+\mu\brak{\hat{i}+\hat{j}-\hat{k}}$\\
            intersect $L_1$ and $L_2$ at $P$ and $Q$ respectively. If $\brak{\alpha,\beta,\gamma}$ is the midpoint of the line segment $PQ$, then $2\brak{\alpha+\beta+\gamma}$ is equal to \dots
            \hfill{[Feb-2024]}
    \item Let $A=\cbrak{1,2,3,\dots,20}$. Let $R_1$ and $R_2$ be two relation on $A$ such that\\
        $R_1=\cbrak{\brak{a,b}\colon b\,is\,divisible\,by\,a}$\\
        $R_2=\cbrak{\brak{a,b}\colon a\,is\,an\,integral\,multiple\,of\,b}$.\\
        Then, the number of elements in $R_1-R_2$ is equal to \dots
        \hfill{[Feb-2024]}
\end{enumerate}
\end{document}
