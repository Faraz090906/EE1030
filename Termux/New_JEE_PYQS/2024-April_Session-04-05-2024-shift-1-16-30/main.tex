%2024-April Session-04-05-2024-shift-1-16-30
\let\negmedspace\undefined
\let\negthickspace\undefined
\documentclass[journal]{IEEEtran}
\usepackage[a5paper, margin=10mm, onecolumn]{geometry}
%\usepackage{lmodern} % Ensure lmodern is loaded for pdflatex
\usepackage{tfrupee} % Include tfrupee package

\setlength{\headheight}{1cm} % Set the height of the header box
\setlength{\headsep}{0mm}     % Set the distance between the header box and the top of the text

\usepackage{gvv-book}
\usepackage{gvv}
\usepackage{cite}
\usepackage{amsmath,amssymb,amsfonts,amsthm}
\usepackage{algorithmic}
\usepackage{graphicx}
\usepackage{textcomp}
\usepackage{xcolor}
\usepackage{txfonts}
\usepackage{listings}
\usepackage{enumitem}
\usepackage{mathtools}
\usepackage{gensymb}
\usepackage{comment}
\usepackage[breaklinks=true]{hyperref}
\usepackage{tkz-euclide} 
\usepackage{listings}
% \usepackage{gvv}                                        
\def\inputGnumericTable{}                                 
\usepackage[latin1]{inputenc}                                
\usepackage{color}                                            
\usepackage{array}                                            
\usepackage{longtable}                                       
\usepackage{calc}                                             
\usepackage{multirow}                                         
\usepackage{hhline}                                           
\usepackage{ifthen}                                           
\usepackage{lscape}

\renewcommand{\thefigure}{\theenumi}
\renewcommand{\thetable}{\theenumi}
\setlength{\intextsep}{10pt} % Space between text and floats


\numberwithin{equation}{enumi}
\numberwithin{figure}{enumi}
\renewcommand{\thetable}{\theenumi}

% Marks the beginning of the document
\begin{document}
\bibliographystyle{IEEEtran}

\title{Assignment 3 \\ 2024-April Session-04-05-2024-shift:1-16-30}
\author{EE24BTECH11049 \\ Patnam Shariq Faraz Muhammed}

% \maketitle
% \newpage
% \bigskip
{\let\newpage\relax\maketitle}

\section*{MCQ}
\begin{enumerate}

    %1st Question
    \item 
    The integral 
    \begin{align*}
        \int_0^{\frac{\pi}{4}}\frac{136\sin{x}}{3\sin{x} + 5\cos{x}}\, dx 
    \end{align*}
    is equal to:

    \hfill{\brak{\text{2024-Apr}}}

    \begin{enumerate}
    \begin{multicols}{2}
        \item $3\pi - 10\log_e\brak{2\sqrt{2}} + 10\log_e5$
        \item $3\pi - 50\log_e2 + 20\log_e5$
        \item $3\pi - 30\log_e2 + 20\log_e5$
        \item $3\pi - 25\log_e2 + 10\log_e5$
    \end{multicols}   
    \end{enumerate}

    %2nd Question 
    \item 
    If $y = y\brak{x}$ is the solution of the differential equation 
    \begin{align*}
        \frac{dy}{dx} + 2y = \sin{\brak{2x}}, y\brak{0} = \frac{3}{4}, \text{ then } y\brak{\frac{\pi}{8}}
    \end{align*}
    is equal to:

    \hfill{\brak{\text{2024-Apr}}}

    \begin{enumerate}
    \begin{multicols}{4}
        \item $e^{\frac{\pi}{8}}$
        \item $e^{-\frac{\pi}{8}}$
        \item $e^{\frac{\pi}{4}}$
        \item $e^{-\frac{\pi}{4}}$
    \end{multicols}   
    \end{enumerate}

    %3rd Question 
    \item 
    Let two straight line drawn from the origin $\vec{O}$ intersect the line $3x + 4y = 12$ at the points $\vec{P}$ and $\vec{Q}$ such that $\Delta OPQ$ is an isosceles triangle and $\angle POQ = 90^{\degree}$. If $l = \vec{OP}^2 + \vec{PQ}^2 + \vec{QO}^2$, then the greatest integer less than or equal to $l$ is:

    \hfill{\brak{\text{2024-Apr}}}

    \begin{enumerate}
    \begin{multicols}{4}
        \item $44$
        \item $48$
        \item $42$
        \item $46$
    \end{multicols}   
    \end{enumerate}

    %4th Question 
    \item 
    If the function 
    \begin{align*}
        f\brak{x} = \frac{\sin{3x} + \alpha\sin{x} - \beta\cos{3x}}{x^3}, x \in \mathbf{R}
    \end{align*}
    is continuous at $x = 0$, then $f\brak{0}$ is equal to:

    \hfill{\brak{\text{2024-Apr}}}

    \begin{enumerate}
    \begin{multicols}{4}
        \item $-4$
        \item $4$
        \item $2$
        \item $-2$
    \end{multicols}   
    \end{enumerate}

    %5th Question 
    onsider the following statements: 
    \begin{align*}
        \textbf{Statement }\mathrm{I:} & \text{For any two complex numbers } z_1, z_2, \\
        & \brak{\abs{z_1} + \abs{z_2}}\abs{\frac{z_1}{\abs{z_1}} + \frac{z_2}{\abs{z_2}}} \leq 2\brak{\abs{z_1} + \abs{z_2}}, \text{ and }\\
        \textbf{Statement }\mathrm{II:} & \text{If } x, y, z \text{ and three distinct complex numbers and } \\
        & a, b, c \text{ are three positive real numbers such that } \\
        & \frac{a}{\abs{y - z}} = \frac{b}{\abs{z - x}} = \frac{c}{\abs{x - y}}, \text{ then} \\
        & \frac{a^2}{y - z} + \frac{b^2}{z - x} + \frac{c^2}{x - y} = 1.
    \end{align*}
    Between the above two statements: 

    \hfill{\brak{\text{2024-Apr}}}

    \begin{enumerate}
        \item statement $\mathrm{I}$ is correct but statement $\mathrm{II}$ is incorrect. 
        \item both statement $\mathrm{I}$ and statement $\mathrm{II}$ are correct.
        \item statement $\mathrm{I}$ is incorrect but statement $\mathrm{II}$ is correct. 
        \item both statement $\mathrm{I}$ and statement $\mathrm{II}$ are incorrect.  
    \end{enumerate}
\end{enumerate}

\section*{INTEGER}
\begin{enumerate}

    %1st Question
    \item
    Let $a_1, a_2, a_3, \dots $ be in arithmetic progression progression of positive terms. Let
    \begin{align*}
        A_k = a_1^2 - a_2^2 + a_3^2 - a_4^2 + \dots + a_{2k-1}^2 -a_{2k}^2.
    \end{align*} 
    If $A_3 = -153$, $A_5 = -435$ and $a_1^2 + a_2^2 + a_3^2 = 66$, then $a_{17} - A_7$ is equal to \rule{1cm}{0.1pt}

    \hfill{\brak{\text{2024-Apr}}}

    %2nd Question 
    \item 
    Suppose $\vec{AB}$ is focal chord of the parabola $y^2 = 12x$ of length $l$ and slope $m < \sqrt{3}$, If the distance of the chord $\vec{AB}$ from the origin is $d$, then $ld^2$ is equal to \rule{1cm}{0.1pt}

    \hfill{\brak{\text{2024-Apr}}}

    %3rd Question 
    \item 
    The number real roots of the equation $\abs{x}\abs{x + 2} -  5\abs{x + 1} - 1 = 0$ is \rule{1cm}{0.1pt}

    \hfill{\brak{\text{2024-Apr}}}

    %4th Question 
    \item 
    If 
    \begin{align*}
        S = \cbrak{a \in \mathbf{R} : \abs{2a - 1} = 3\sbrak{a} + 2\cbrak{a}}, A = 72\sum_{a \in S}a, 
    \end{align*}
    where $\sbrak{t}$ denotes the greatest integer less than or equal to $t$ and $\cbrak{t}$ represents the fractional part of $t$, then $A$ is equal to \rule{1cm}{0.1pt}

    \hfill{\brak{\text{2024-Apr}}}

    %5th Question 
    \item 
    Let $\vec{\Bar{a}} = \hat{i} - \vec{3}\hat{j} + \vec{7}\hat{k}$, $\vec{\Bar{b}} = \vec{2}\hat{i} - \hat{j} + \hat{k}$ and $\vec{\Bar{c}}$ be a vector such that $\brak{\vec{\Bar{a}} + 2\vec{\Bar{b}}} \times \vec{\Bar{c}} = 3\brak{\vec{\Bar{c}} \times \vec{\Bar{a}}}$. If $\vec{\Bar{a}} . \vec{\Bar{c}} = 130$, then $\vec{\Bar{b}} . \vec{\Bar{c}}$ is equal to \rule{1cm}{0.1pt}

    \hfill{\brak{\text{2024-Apr}}}
    
    %6th Question 
    \item 
    The area of the  region enclosed by the parabolas $y = x^2 - 5x$ and $y = 7x - x^2$ is \rule{1cm}{0.1pt}

    \hfill{\brak{\text{2024-Apr}}}

    %7th Question 
    \item 
    let $f$ be a differentiable function in the interval $\brak{0, \infty}$ such that $f\brak{1} = 1$ and 
    \begin{align*}
        \lim_{t \to x} \frac{t^2f\brak{x} - x^2f\brak{t}}{t - x} = 1 \text{ for each } x > 0.
    \end{align*}
    Then $2f\brak{2} + 3f\brak{3}$ is equal to \rule{1cm}{0.1pt}

    \hfill{\brak{\text{2024-Apr}}}

    %8th Question 
    \item 
    If the constant term in the expansion of 
    \begin{align*}
        \brak{1 + 2x - 3x^3}\brak{\frac{3}{2}x^2 - \frac{1}{3x}}^9 \text{ is } p,
    \end{align*}
    then $108p$ is equal to \rule{1cm}{0.1pt}

    \hfill{\brak{\text{2024-Apr}}}

    %9th Question 
    \item 
    From a lot of $10$ items, which include $3$ defective items, a sample of $5$ items is drawn at random. Let the random variable $X$ denote the numbers of defective items in the sample. If the variance of $X$ is $\sigma$, then $96\sigma^2$ is equal to \rule{1cm}{0.1pt}

    \hfill{\brak{\text{2024-Apr}}}

    %10th Question 
    \item 
    The number of ways of getting sum $16$ on throwing a dice four times is \rule{1cm}{0.1pt}

    \hfill{\brak{\text{2024-Apr}}}

\end{enumerate}

\end{document}C
