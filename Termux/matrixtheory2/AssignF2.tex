%iffalse
\let\negmedspace\undefined
\let\negthickspace\undefined
\documentclass[journal,12pt,twocolumn]{IEEEtran}
\usepackage{cite}
\usepackage{amsmath,amssymb,amsfonts,amsthm}
\usepackage{algorithmic}
\usepackage{graphicx}
\usepackage{textcomp}
\usepackage{xcolor}
\usepackage{txfonts}
\usepackage{listings}
\usepackage{enumitem}
\usepackage{mathtools}
\usepackage{gensymb}
\usepackage{comment}
\usepackage[breaklinks=true]{hyperref}
\usepackage{tkz-euclide} 
\usepackage{listings}
\usepackage{gvv}                                        
%\def\inputGnumericTable{}                                 
\usepackage[latin1]{inputenc}                                
\usepackage{color}                                            
\usepackage{array}                                            
\usepackage{longtable}                                       
\usepackage{calc}                                             
\usepackage{multirow}                                         
\usepackage{hhline}                                           
\usepackage{ifthen}                                           
\usepackage{lscape}
\usepackage{tabularx}
\usepackage{array}
\usepackage{float}
\usepackage{ulem}


\newtheorem{theorem}{Theorem}[section]
\newtheorem{problem}{Problem}
\newtheorem{proposition}{Proposition}[section]
\newtheorem{lemma}{Lemma}[section]
\newtheorem{corollary}[theorem]{Corollary}
\newtheorem{example}{Example}[section]
\newtheorem{definition}[problem]{Definition}
\newcommand{\BEQA}{\begin{eqnarray}}
\newcommand{\EEQA}{\end{eqnarray}}
\newcommand{\define}{\stackrel{\triangle}{=}}
\theoremstyle{remark}
\newtheorem{rem}{Remark}
\renewcommand{\thefigure}{\theenumi}
\renewcommand{\thetable}{\theenumi}

% Marks the beginning of the document
\begin{document}
\bibliographystyle{IEEEtran}

\title{{\uline{Assignment 2 \\ } \\}Chapter-12: \\Differentiation}
\author{{EE24BTECH11049 \\ Patnam Shariq Faraz Muhammed}}

\maketitle
\newpage
\bigskip

\begin{enumerate}

%1st question
	\item
	{If $y$=$\brak{x+\sqrt{1+x^{2}}}^{n}$,then $\brak{1+x^{2}}.\frac{d^{2}y}{dx^{2}} + x.\frac{dy}{dx}$ is}

	\hfill{\sbrak{2002}}

	\begin{enumerate}[label=(\alph*)]
		\item $n^{2}.y$
		\item $-n^{2}.y$
		\item $-y$
		\item $2.x^{2}.y$
	\end{enumerate}

%2nd question
	\item
	If $f\brak{y}=e^{y}$, $g\brak{y}=y; y>0$ and $F\brak{t}= \int_{0}^{t}{f\brak{t-y}.g\brak{y} \, dt}$, then

	\hfill{\sbrak{2003}}

	\begin{enumerate}[label=(\alph*)]
		\item $F\brak(t)=t.e^{-t}$
		\item $F\brak(t)=1-t.e^{-t}.\brak{1+t}$
		\item $e^{t}-\brak{1+t}$
		\item $F\brak(t)=t.e^{t}$
	\end{enumerate}

%3rd question
	\item 
		If $f\brak{x}=x^{n}$, then the value of $f\brak{1}-\frac{f'\brak{1}}{1!}+\frac{{f''\brak{1}}}{2!}-\frac{{f'''\brak{1}}}{3!}+\dots \frac{\brak{-1}^n.f^{n}\brak{1}}{n}$ is

	\hfill{\sbrak{2003}}
	
	\begin{enumerate}[label=(\alph*)]
		\item $1$
		\item $2^{n}$
		\item $2^{n}-1$
		\item $0$
	\end{enumerate}
	
\end{enumerate}


\end{document}

