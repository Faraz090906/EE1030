%iffalse
\let\negmedspace\undefined
\let\negthickspace\undefined
\documentclass[journal,12pt,twocolumn]{IEEEtran}
\usepackage{cite}
\usepackage{amsmath,amssymb,amsfonts,amsthm}
\usepackage{algorithmic}
\usepackage{graphicx}
\usepackage{textcomp}
\usepackage{xcolor}
\usepackage{txfonts}
\usepackage{listings}
\usepackage{enumitem}
\usepackage{mathtools}
\usepackage{gensymb}
\usepackage{comment}
\usepackage[breaklinks=true]{hyperref}
\usepackage{tkz-euclide} 
\usepackage{listings}
\usepackage{gvv}                                        
%\def\inputGnumericTable{}                                 
\usepackage[latin1]{inputenc}                                
\usepackage{color}                                            
\usepackage{array}                                            
\usepackage{longtable}                                       
\usepackage{calc}                                             
\usepackage{multirow}                                         
\usepackage{hhline}                                           
\usepackage{ifthen}                                           
\usepackage{lscape}
\usepackage{tabularx}
\usepackage{array}
\usepackage{float}
\usepackage{ulem}


\newtheorem{theorem}{Theorem}[section]
\newtheorem{problem}{Problem}
\newtheorem{proposition}{Proposition}[section]
\newtheorem{lemma}{Lemma}[section]
\newtheorem{corollary}[theorem]{Corollary}
\newtheorem{example}{Example}[section]
\newtheorem{definition}[problem]{Definition}
\newcommand{\BEQA}{\begin{eqnarray}}
\newcommand{\EEQA}{\end{eqnarray}}
\newcommand{\define}{\stackrel{\triangle}{=}}
\theoremstyle{remark}
\newtheorem{rem}{Remark}

% Marks the beginning of the document
\begin{document}
\bibliographystyle{IEEEtran}

\title{{\uline{Assignment 1 \\ } \\}Chapter-11: \\Limits, Continuity and Differentiability}
\author{{EE24BTECH11049 \\ Patnam Shariq Faraz Muhammed}}


 
\maketitle
\newpage
\bigskip

\renewcommand{\thefigure}{\theenumi}
\renewcommand{\thetable}{\theenumi}

\section*{D \brak{11-25}: MCQs with One or More thanOne correct}

\begin{enumerate}
    \item 
%1st question
    {Let $g\brak{x}=x.f\brak{x}$, where\\[6pt] $f\brak{x}$ = $\begin{cases}
        x.\sin\frac{1}{x}, & x\neq 0\\
        0, & x=0
    \end{cases}$
    . At $x=0$} \\
    
    \hfill 
    {\brak{1995}}
    
    \begin{enumerate}[label=(\alph*)]
        
        \item $g$ is differentiable but $g'$ is not continuous
        \item $g$ is differentiable while $f$ is not
        \item both $f$ and $g$ are differentiable
        \item $g$ is differentiable and $g'$ is continuous 
    \end{enumerate}

    \item 
%2nd question
     {The function $f\brak{x}$ = max$\{\brak{1-x},\brak{1+x},2\}$, $x\in \brak{-\infty,\infty}$} \\

    \hfill 
    {\brak{1995}}
    
    \begin{enumerate}[label=(\alph*)]
        
        \item continuous at all points
        \item differentiable at all points
        \item differentiable at all points except at $x=l$ and $x=-1$
        \item continuous at all points except at $x=l$ and $x=-1$ where it is discontinuous
    \end{enumerate}


    \item 
%3rd question
    Let $h\brak{x}$ = min$\{x,x^{2}\}$ \\

    \hfill 
    {\brak{1998-2 marks}}

    \begin{enumerate}[label=(\alph*)]
        
        \item $h$ is continuous for all $x$
        \item $h$ is differentiable for all $x$
        \item $h'\brak{t}=1$, for all $x >1$
        \item $h$ is not differentiable at two values of $x$
    \end{enumerate}


    \item 
%4th question
    {$\lim_{x \to1}\frac{\sqrt{1-\cos 2\brak{x-1}}}{x-1}$} \\

    \hfill 
    {\brak{1998-2 marks}}
    \begin{enumerate}[label=(\alph*)]
        
        \item exists and it equals $\sqrt{2}$
        \item exists and it equals $-\sqrt{2}$
        \item does not exist because $x-1\mapsto 0$
        \item does not exist because the left-hand limit is not equal to the right-hand limit
    \end{enumerate}


    \item 
%5th question
    {If $f\brak{x}$ = min $\{1,x^2,x^3\}$} \\

    \hfill 
    {\brak{2006, 5M, -1}}
    
    \begin{enumerate}[label=(\alph*)]
        
        \item $f\brak{x}$ is continuous $\forall x \in R$
        \item $f\brak{x}$ is continuous and differentiable everywhere
        \item $f\brak{x}$ is not diferentiable at two points
        \item $f\brak{x}$ is not differentiable at one point
    \end{enumerate}


    \item 
%6th question
    {Let $L=\lim_{x \to0}\frac{a-\sqrt{a^2-x^2-\frac{x^2}{4}}}{x^4}, a>0$.\\ If L is finite, then} \\

    \hfill 
    {\brak{2009}}
    
    \begin{enumerate}[label=(\alph*)]
        
        \item $a=2$ 
        \item $a=1$
        \item $L=\frac{1}{64}$
        \item $L=\frac{1}{32}$
    \end{enumerate}


    \item 
%7th question
    {Let $f:R \to R$ be a function such that $f\brak{x+y}=f\brak{x}+f\brak{y}$, $\forall x,y\in R$. If $f\brak{x}$ is differentiable at $x=0$, then} 

    \hfill 
    {\brak{2011}}
    
    \begin{enumerate}[label=(\alph*)]
        
        \item $f\brak{x}$ is differentiable only in a finite interval containing zero 
        \item $f\brak{x}$ is continuous $\forall x\in R$
        \item $f'\brak{x}$ is constant $\forall x\in R$
        \item $f\brak{x}$ is differentiable except at finitely many points 
    \end{enumerate}


    \item 
%8th question
    {If $f\brak{x}$= 
    $\begin{cases}
        -x-\frac{\pi}{2}, & x\leq \frac{\pi}{2} \\
        -\cos x, & \frac{\pi}{2}<x\leq 0 \\
        x-1, & 0<x\leq1 \\
        \ln x, & x>1
    \end{cases}$} 

    \hfill 
    {\brak{2011}}
    
    \begin{enumerate}[label=(\alph*)]
        
        \item $f\brak{x}$ is continuous at $x=\frac{\pi}{2}$
        \item $f\brak{x}$ is not differentiable at $x=0$
        \item $f'\brak{x}$ is differentiable at $x=1$
        \item $f\brak{x}$ is differentiable at $x=\frac{3}{2}$
    \end{enumerate}


    \item 
%9th question
    {For every integer $n$, let $a_n$ and $b_n$, be real numbers. Let function $f\brak{x}: IR \mapsto IR$ be given by
    $f\brak{x}$= 
    $\begin{cases}
       a_n+\sin\pi x, & for x\in [2n,2n+1] \\
       b_n+\cos\pi x, & for x\in \brak{2n-1,2n}
    \end{cases}$
    for all integers $n$. If $f$ is continuous,then which of the following hold\brak{s} for all $n$} 

    \hfill 
    {\brak{2012}}
    
    \begin{enumerate}[label=(\alph*)]
        
        \item $a_{n-1}-b_{n-1}=0$ 
        \item $a_n-b_n=1$ 
        \item $a_n-b_{n+1}=1$ 
        \item $a_{n-1}-b_n=-1$ 
    \end{enumerate}


    \item 
%10th question
    {For $a\in R$ \brak{the set of all real numbers}, $a\neq -1$ 
    $\lim_{n\to\infty}\frac{\brak{1^a+2^a+\cdots+n^a}}{\brak{n+1}^a.[\brak{na+1}+\brak{na+2}+\cdots+\brak{na+n}]}$ = $\frac{1}{60}$ Then $a=$ }

    \hfill 
    {\brak{\textit{JEE Adv.}2013}}
    
    \begin{enumerate}[label=(\alph*)]
        
        \item $5$
        \item $7$ 
        \item $\frac{-15}{2}$ 
        \item $\frac{-17}{2}$ 
    \end{enumerate}


    \item 
%11th question
    {Let $f: [a,b]\mapsto [1,\infty)$ be a continuous function and let $g: R\mapsto R$ be defined as 
    $f\brak{x}$= 
    $\begin{cases}
       0, & if x<a, \\
       \displaystyle \int_{a}^{x}{f\brak{t} \, dt}, & if a\leq x\leq b \\
       \displaystyle \int_{a}^{b}{f\brak{t} \, dt}, & if x>b
    \end{cases}$; then} 

    \hfill 
    {\brak{\textit{JEE Adv.}2013}}
    
    \begin{enumerate}[label=(\alph*)]
        
        \item $g\brak{x}$ is continuous but not differentiable at $a$
        \item $g\brak{x}$ is differentiable on $R$
        \item $g\brak{x}$ is continuous but not differentiable at $b$
        \item $g\brak{x}$ is continuous and differentiable at either $\brak{a}$ or $\brak{b}$ but not both 
    \end{enumerate}


    \item 
%12th question
    {For every pair of continuous functions $f, g: [0, 1]\mapsto R$ such that max $\{f\brak{x}: x\in [0, 1]\}$= max $\{g\brak{x}: x\in [0, 1]\}$, the correct statement\brak{s} is\brak{are}:}
   
    \hfill 
    {\brak{\textit{JEE Adv.}2014}}
    
    \begin{enumerate}[label=(\alph*)]
        
        \item ${\brak{f\brak{c}}}^2+3.f\brak{c}={\brak{g\brak{c}}}^2+3.g\brak{c}$ for some $c\in [0,1]$
        \item ${\brak{f\brak{c}}}^2+f\brak{c}={\brak{g\brak{c}}}^2+3.g\brak{c}$ for some $c\in [0,1]$
        \item ${\brak{f\brak{c}}}^2+3.f\brak{c}={\brak{g\brak{c}}}^2+g\brak{c}$ for some $c\in [0,1]$
        \item ${\brak{f\brak{c}}}^2={\brak{g\brak{c}}}^2$ for some $c\in [0,1]$ 
    \end{enumerate}


    \item 
%13th question    
    {Let $g: R\mapsto R$ be a differentiable function with $g\brak{0}=0$, $g'\brak{0}=0$ and $g'\brak{1}\neq 0$. Let $f\brak{x}=
        \begin{cases}
            \frac{x}{|x|}.g\brak{x}, & x\neq 0 \\
            0, & x=0
        \end{cases}$ 
        and $h\brak{x}=e^{|x|}$ for all $x\in R$. Let $\brak{f\circ h}\brak{x}$ denote $f\brak{h\brak{x}})$ and $\brak{h\circ f}\brak{x}$ denote $h\brak{f\brak{x}}$. Then which of the following is\brak{are} true?}
        
    \hfill 
    {\brak{\textit{JEE Adv.}2015}}
    
    \begin{enumerate}[label=(\alph*)]
        
        \item $f$ is differentiable at $x=0$ 
        \item $h$ is differentiable at $x=0$ 
        \item $f\circ h$ is differentiable at $x=0$ 
        \item $h\circ f$ is differentiable at $x=0$  
    \end{enumerate}


    \item 
%14th question
    {Let $a, b\in R$ and $f: R\mapsto R$ be defined by $f\brak{x}=a.\cos \brak{|x^3-x|} +b.|x|.\sin \brak{|x^3+x|}$. Then $f$ is}   
        
    \hfill 
    {\brak{\textit{JEE Adv.}2016}}
    
    \begin{enumerate}[label=(\alph*)]
        
        \item differentiable at $x=0$ if $a=0$ and $b=1$
        \item differentiable at $x=1$ if $a=1$ and $b=0$
        \item {NOT} differentiable at $x=0$ if $a=1$ and $b=0$
        \item {NOT} differentiable at $x=1$ if $a=0$ and $b=1$
    \end{enumerate}


    \item 
%15th question
    {Let $f:[-\frac{1}{2}, 2]\mapsto R$ and $g:[-\frac{1}{2}, 2]\mapsto R$ be functions defined by $f\brak{x}=[x^2-3]$ and $g\brak{x}=|x|.f\brak{x}+|4x-7|.f\brak{x}$, where $[y]$ denotes the greatest integer less than or equal to $y$ for $y\in R$. Then}   
        
    \hfill 
    {\brak{\textit{JEE Adv.}2016}}
    
    \begin{enumerate}[label=(\alph*)]
        
        \item $f$ is discontinuous exactly at three points in $[-\frac{1}{2}, 2]$
        \item $f$ is discontinuous exactly at four points in $[-\frac{1}{2}, 2]$
        \item $g$ is NOT differentiable exactly at four points in $[-\frac{1}{2}, 2]$
        \item $g$ is NOT differentiable exactly at five points in $[-\frac{1}{2}, 2]$
    \end{enumerate}

\end{enumerate}

\end{document} 
