\let\negmedspace\undefinedx
\let\negthickspace\undefined
\documentclass[journal]{IEEEtran}
\usepackage[a5paper, margin=10mm, onecolumn]{geometry}
%\usepackage{lmodern} % Ensure lmodern is loaded for pdflatex
\usepackage{tfrupee} % Include tfrupee package

\setlength{\headheight}{1cm} % Set the height of the header box
\setlength{\headsep}{0mm}     % Set the distance between the header box and the top of the text

\usepackage{gvv-book}
\usepackage{gvv}
\usepackage{cite}
\usepackage{amsmath,amssymb,amsfonts,amsthm}
\usepackage{algorithmic}
\usepackage{graphicx}
\usepackage{textcomp}
\usepackage{xcolor}
\usepackage{txfonts}
\usepackage{listings}
\usepackage{enumitem}
\usepackage{mathtools}
\usepackage{gensymb}
\usepackage{comment}
\usepackage[breaklinks=true]{hyperref}
\usepackage{tkz-euclide} 
\usepackage{listings}
% \usepackage{gvv}                                        
\def\inputGnumericTable{}                                 
\usepackage[latin1]{inputenc}                                
\usepackage{color}                                            
\usepackage{array}                                            
\usepackage{longtable}                                       
\usepackage{calc}                                             
\usepackage{multirow}                                         
\usepackage{hhline}                                           
\usepackage{ifthen}                                           
\usepackage{lscape}

\renewcommand{\thefigure}{\theenumi}
\renewcommand{\thetable}{\theenumi}
\setlength{\intextsep}{10pt} % Space between text and floats


\numberwithin{equation}{enumi}
\numberwithin{figure}{enumi}
\renewcommand{\thetable}{\theenumi}

% Marks the beginning of the document
\begin{document}
\bibliographystyle{IEEEtran}

\title{Assignment 5}
\author{EE24BTECH11049 \\ Patnam Shariq Faraz Muhammed}

% \maketitle
% \newpage
% \bigskip
{\let\newpage\relax\maketitle}
\section{Carry TWO marks Each}
\begin{enumerate}
		%1st Question 
	\item A student performed X-rays diffraction experiment on a FCC polycrystalline pure metal. The following $\sin^{2}{\theta}$ values were calculated from the diffraction peaks. 
		\begin{align*}
			\sin^{2}{\theta} = 0.136, 0.185, 0.504, 0.544
		\end{align*}
	However, the student was negligent and missed noting one of the peaks. Which one of the following Miller indices corresponds to the missing peak? 
		
		\begin{multicols}{4}
			\begin{enumerate}
				\item $\brak{200}$
				\item $\brak{220}$
				\item $\brak{311}$
				\item $\brak{222}$
			\end{enumerate}
		\end{multicols}

		%2nd Question 
	\item Match the lattice planes and directions \brak{\text{in Column }\mathrm{I}} with the corresponding Miller indices \brak{\text{in Column }\mathrm{I}}
		\begin{multicols}{2}
			\textbf{Column} $\mathrm{I}$
			\begin{enumerate}
				\item[P]
				\item[Q]
				\item[R]
				\item[S]
			\end{enumerate}
			\columnbreak
			\textbf{Column} $\mathrm{II}$
			\begin{enumerate}
				\item[1] $\brak{\hat{1}11}$
				\item[2] $\brak{\hat{1}12}$
				\item[3] $\brak{\hat{2}21}$
				\item[4] $\brak{\hat{1}10}$
			\end{enumerate}
		\end{multicols}
		\begin{enumerate}
			\item P-2, Q-4, R-1, S-3 
			\item P-3, Q-1, R-4, S-2
			\item P-2, Q-4, R-3, S-1
			\item P-3, Q-4, R-2, S-1
		\end{enumerate}

		%3rd Question 
	\item Match the hardness test \brak{\text{in Column }\mathrm{I}} with its indenter type \brak{\text{in Column }\mathrm{II}}
		\begin{multicols}{2}
			\textbf{Column} $\mathrm{I}$
			\begin{enumerate}
				\item [P] Brinell
				\item [Q] Rockwell
				\item [R] Vickers
			\end{enumerate}
			\columnbreak
			\textbf{Column} $\mathrm{II}$
			\begin{enumerate}
				\item[1] Diamond pyramidal
				\item[2] Diamond cone
				\item[3] Tungsten carbide sphere
				\item[4] Steel sphere
			\end{enumerate}
		\end{multicols}

		\begin{enumerate}
			\item P-2, Q-4, R-1
			\item P-4, Q-2, R-3
			\item P-3, Q-4, R-2
			\item P-4, Q-2, R-1
		\end{enumerate}

		%4th Question 
	\item TTT diagram of a eutectoid steel is shown below. Match the heat treatment cycle \brak{\text{in Column }\mathrm{I}} with its microstructure \brak{\text{in Column }\mathrm{II}}


		\begin{multicols}{2}
			\textbf{Column} $\mathrm{I}$
			\begin{enumerate}
				\item P
				\item Q
				\item R
			\end{enumerate}
			\columnbreak
                        \textbf{Column} $\mathrm{II}$
			\begin{enumerate}
				\item[1] Bainite only
				\item[2] Pearlite only
				\item[3] Pearlite + Bainite + Martensite
				\item[4] Pearlite + Martensite
			\end{enumerate}
		\end{multicols}

		\begin{enumerate}
			\item P-1, Q-2, R-4
			\item P-2, Q-3, R-2
			\item P-2, Q-4, R-1
			\item P-2, Q-3, R-1
		\end{enumerate}

		%5th Question
	\item Which of the following statement\brak{\text{s}} is/are true for an optical microscope?
		
		\begin{enumerate}
			\item Increasing the aperture of the objective lens deteriorates the resolution
			\item Reducing the wavelength of illuminating light improves the resolution
			\item Increasing the refractive index of the medium in between the sample and the objective lens improves the resolution
			\item Reducing the wavelength of illuminating light decreases the depth of field
		\end{enumerate}

		%6th Question 
	\item Among the 14 Bravais lattices, there is no base centred cubic unit cell. Which of the following statement\brak{\text{s}} is/are true?
		\begin{enumerate}
			\item The base-centred cubic unit cell is same as the simple tetragonal unit cell
			\item The base-centred cubic unit cell is same as the body centred tetragonal unit cell
			\item The base-centred cubic unit cell is same as the simple orthorhombic unit cell
			\item The base-centred cubic unit cell does not have any 3-fold rotation axis
		\end{enumerate}

		%7th Question 
	\item Specific heat $\brak{C_v}$ of a material was found to depend on temperature as shown below. Which of the following statement\brak{\text{s}} is/are true 
		\begin{enumerate}
			\item The material is metallic
			\item The material is insulating
			\item The material is three dimensional
			\item The material is one dimensional
		\end{enumerate}

		%8th Question 
	\item A pure Silicon wafer is doped with Boron by exposing it to $B_2O_3$ vapour at an elevated temperature. It takes 1000 seconds to reach a Boron concentration of $10^{20} atoms-m^{-3}$ at a depth of $1\mu m$ is \brak{\text{in seconds}}:\rule{1cm}{0.1pt} \brak{\text{rounded
off to nearest integer}}
	Given: Boron concentration on the wafer surface remains constant.

		%9th Question

\end{document}
