\let\negmedspace\undefinedx
\let\negthickspace\undefined
\documentclass[journal]{IEEEtran}
\usepackage[a5paper, margin=10mm, onecolumn]{geometry}
%\usepackage{lmodern} % Ensure lmodern is loaded for pdflatex
\usepackage{tfrupee} % Include tfrupee package

\setlength{\headheight}{1cm} % Set the height of the header box
\setlength{\headsep}{0mm}     % Set the distance between the header box and the top of the text

\usepackage{gvv-book}
\usepackage{gvv}
\usepackage{cite}
\usepackage{amsmath,amssymb,amsfonts,amsthm}
\usepackage{algorithmic}
\usepackage{graphicx}
\usepackage{textcomp}
\usepackage{xcolor}
\usepackage{txfonts}
\usepackage{listings}
\usepackage{enumitem}
\usepackage{mathtools}
\usepackage{gensymb}
\usepackage{comment}
\usepackage[breaklinks=true]{hyperref}
\usepackage{tkz-euclide} 
\usepackage{listings}
% \usepackage{gvv}                                        
\def\inputGnumericTable{}                                 
\usepackage[latin1]{inputenc}                                
\usepackage{color}                                            
\usepackage{array}                                            
\usepackage{longtable}                                       
\usepackage{calc}                                             
\usepackage{multirow}                                         
\usepackage{hhline}                                           
\usepackage{ifthen}                                           
\usepackage{lscape}

\renewcommand{\thefigure}{\theenumi}
\renewcommand{\thetable}{\theenumi}
\setlength{\intextsep}{10pt} % Space between text and floats


\numberwithin{equation}{enumi}
\numberwithin{figure}{enumi}
\renewcommand{\thetable}{\theenumi}

% Marks the beginning of the document
\begin{document}
\bibliographystyle{IEEEtran}

\title{Assignment 4}
\author{EE24BTECH11049 \\ Patnam Shariq Faraz Muhammed}

% \maketitle
% \newpage
% \bigskip
{\let\newpage\relax\maketitle}

\section{MA: Mathematics}
\subsection{Carry one mark each}
	\begin{enumerate}
		%1st Question
		\item 
		If $\brak{x_1^*, x_2^*}$ is an optional solution of the linear programming problem, minimize $x_1 + 2x_2$ subject to 
		\begin{align*}
			4x_1 - x_2 & \geq 8\\
			2x_1 + x_2 & \geq 10\\
			-x_1 + x_2 & \leq 7\\
			x_1, x_2 & \geq 0
		\end{align*}
		and \brak{\lambda_1^*, \lambda_2^*, \lambda_3^*} is an optional solution of its dual problem, then $\sum_{i = 1}^2 x_i^*0 + \sum_{j = 1}^3 \lambda_j^*$ is equal to \rule{2cm}{0.1pt} \brak{\text{correct upto one decimal place}}

		%2nd Question 
		\item 
		let $a, b, c \in \mathbf{R}$ be such that the quadrature rule 
		\begin{align*}
			\int_{-1}^1 f\brak{x}\, dx \approx af\brak{-1} + bf\brak{0} + cf^\prime\brak{1} 
		\end{align*}

		is exact for all polynomials of degree less than or equal to 2. Then $b$ is equal to \rule{2cm}{0.1pt} \brak{\text{rounded off to two decimal places}}

		%3rd Question 
		\item 
		Let $f\brak{x} = x^4$ and let $p\brak{x}$ be the interpolating polynomial of $f$ at nodes 1,2 and 3. Then $p\brak{0}$ is equal to \rule{2cm}{0.1pt}

		%4th Question
		\item
		For $n \geq 2$, define the sequece $\cbrak{x_n}$ by 
		\begin{align*} 
			x_n = \frac{1}{2\pi}\int_0^{\frac{\pi}{2}} tan^{\frac{1}{n}}{t}\, dt. 
		\end{align*}
		Then the sequence $\cbrak{x_n}$ converges to \rule{2cm}{0.1pt}\\ \brak{\text{correct up to two decimal places}}

		%5th Question 
		\item 
		\begin{align*}
			L^2\sbrak{0,10} = \cbrak{f:\sbrak{0,10}\mapsto \mathbf{R}: f \text{ is lebesgue measurable and }\int_0^{10}f^2dx < \infty}
		\end{align*}
		equipped with the norm $\norm{f} = \brak{\int_0^{10} f^2\, dx}^{\frac{1}{2}}$ and let $T$ be the linear functional on $L^2\sbrak{0,10}$ given by 
		\begin{align*}
			T\brak{f} = \int_0^2 f\brak{x}\, dx - \int_3^{10} f\brak{x}\, dx.
		\end{align*}
		Then $\norm{T}$ is equal to \rule{2cm}{0.1pt}

		%6th Question 
		\item 
		if $\cbrak{x_{13}, x_{22}, x_{23} = 10, x_{31}, x_{32}, x_{34}}$ is the set of basic variable of  balanced transportation problem seeking to minimize cost of transportation from origins to destinations, where the cost matrix is, 
		\begin{table}
			\centering
			\begin{tabular}{|c|c|c|c|c|c|}
\hline 
 & $D_1$ & $D_2$ & $D_3$ & $D_4$ & Availability\\
\hline
$O_1$ & 6 & 2 & -1 & 0 & 10 \\
\hline 
$O_2$ & 4 & 2 & 2 & 3 & $\lambda + 5$\\
\hline 
$O_3$ & 3 & 1 & 2 & 1 & $3\lambda$\\
\hline 
Demand & 10 & $\mu - 5$ & $\mu + 5$ & 15 & \\
\hline
\end{tabular}

		\end{table}
		and $\lambda, \mu \in \mathbf{R}$ is equal to \rule{2cm}{1pt}

		%7th Question 
		\item 
		Let $\mathbf{Z}_{225}$ be the ring of integers modulo 225. If $x$ is the number of prime ideals and $y$ is the number of non trivial units $\mathbf{Z}_{225}$, then $x + y$ is equal to \rule{2cm}{0.1pt}

		%8th Question 
		\item 
		let $u\brak{x,t}$ be the solution of
		\begin{align*}
		\frac{\partial^2 u}{\partial t^2} - \frac{\partial^2 u}{\partial x^2} = 0, u\brak{x,0} = f\brak{x}, \frac{\partial u}{\partial t}\brak{x,0} = 0, x\in \mathbf{R}, t > 0,
		\end{align*}
		Where $f$ is twice continuously differentiable function. If $f\brak{-2} = 4, f\brak{0} = 0$ and $u\brak{2,2} = 8$, then each value of $u\brak{1,3}$ is \rule{2cm}{0.1pt}

\subsection{Carry two marks each}
		%9th Question 
		\item 
		Let $\cbrak{e_n}_{n = 1}^{\infty}$ be an orthomormal basis for a separable Hilbert space $H$ with the inner product $\langle \cdot, \cdot \rangle$. Define 
		\begin{align*}
			f_n = e_n - \frac{1}{n + 1}e_{n + 1} \text{ for } n \in \mathbf{N}
		\end{align*}
		Then 
		\begin{enumerate}
			\item the closure of the span $\cbrak{f_n : n \in \mathbf{N}}$ equals $H$
			\item $f = 0$ if $\langle{f}, f_n \rangle = \langle f, e_n \rangle$ for all $ n \in \mathbf{N}$
			\item $\cbrak{f_n}_{n = 1}^{\infty}$ is an orthogonal subset of $H$
			\item there does not exist non zero $f \in H$ such that $\langle f , e_2\rangle = \langle{f}, f_2\rangle$
		\end{enumerate}

		%10th Question 
		\item
			Suppose $V$ is finite dimensional non-zero vector space over $\mathbf{C}$ and $T: V \mapsto V$ is a linear transformation such that Range$\brak{T}$ = Nullspace$\brak{T}$. Then which of following statements is FALSE?
		
		\begin{enumerate}
			\item The dimensions of $V$ is even 
			\item 0 is the only eigenvalue of $T$
			\item Both 0 and 1 are the eigen values of $T$
			\item $T^2 = 0$
		\end{enumerate}

		%11th Question 
		\item 
		Let $P \in M_{m \times n}\brak{R}$. Consider the following statements:
		\begin{tabular}{c  c}
			$mathrm{I:}$ & If $XPY = 0$ for all $X \in M_{1 \times m}\brak{\mathbf{R}}$, then $P = 0$.\\
			$mathrm{II:}$ & If $m = n$, $P$ is symmetric and $P^2 = 0$, then $P = 0i$. \\
		\end{tabular}
		Then 
		\begin{enumerate}
			\begin{multicols}{2}
			\item both $\mathrm{I}$ and $\mathrm{II}$ are true
			\item $\mathrm{I}$ is true but $\mathrm{II}$ is false 
			\item $\mathrm{I}$ is false but $\mathrm{II}$ is true
			\item both $\mathrm{I}$ and $\mathrm{II}$ are false
			\end{multicols}
		\end{enumerate}

		%12th Question
		\item 
			For $n \in \mathbf{N}$, let $T_n: \brak{l^1,\norm{\cdot}_1} \mapsto \brak{l^{\infty}, \norm{\cdot}_{\infty}}$ and $T: \brak{l^1,\norm{\cdot}_1} \mapsto \brak{l^{\infty}, \norm{\cdot}_{\infty}}$ be the bounded linear operators defined by 
		\begin{align*}
			T_n\brak{x_1,x_2,\dots} = \brak{y_1, y_2, \dots}, \text{ where } y_j = 
			\begin{cases}
				x_j ,& j \leq n\\
				x_n ,& j > n
			\end{cases}\\
		\end{align*}
		and 
		\begin{align*}
			T\brak{x_1, x_2, \dots} = \brak{x_1, x_2, \dots}
		\end{align*}
		Then 
		\begin{enumerate}
			\item $\norm{T_n}$ does not converge to $\norm{T}$ as $n\to \infty$
			\item $\norm{T_n - T}$ converges to zero as $n\to \infty$
			\item for all $x \in l^1$, $\norm{T_n\brak{x} - T\brak{x}}$ converges to zero as $n\to \infty$
			\item for each non-zero $x \in l^1$, there exists a continuous linear functional $g$ on $l^{\infty}$ such that $g\brak{T_n\brak{x}}$ does not converge to $g\brak{T\brak{x}}$ as $n\to \infty$
		\end{enumerate}

		%13th Question 
		\item 
		Let $P\brak{\mathbf{R}}$ denote the power set of $\mathbf{R}$, equipped with the metric 
		\begin{align*}
			d\brak{U,V} = \text{sup}_{x \in \mathbf{R}}\abs{\chi_U\brak{x} - \chi_V\brak{x}},
		\end{align*}
		where $\chi_U$ and $\chi_V$ denote the characteristic function of subsets $U$ and $V$, respectively of $\mathbf{R}$. The set $\cbrak{\cbrak{m}:m \in \mathbf{Z}}$ in the metric space $\brak{P\brak{\mathbf{R}}, d}$ is 
		\begin{multicols}{2}
			\begin{enumerate}
				\item bounded but not totally bounded 
				\item totally bounded but not compact
				\item compact 
				\item not bounded
			\end{enumerate}
		\end{multicols}
	\end{enumerate}
\end{document}
