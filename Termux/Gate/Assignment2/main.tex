\let\negmedspace\undefined
\let\negthickspace\undefined
\documentclass[journal]{IEEEtran}
\usepackage[a5paper, margin=10mm, onecolumn]{geometry}
%\usepackage{lmodern} % Ensure lmodern is loaded for pdflatex
\usepackage{tfrupee} % Include tfrupee package

\setlength{\headheight}{1cm} % Set the height of the header box
\setlength{\headsep}{0mm}     % Set the distance between the header box and the top of the text

\usepackage{gvv-book}
\usepackage{gvv}
\usepackage{cite}
\usepackage{amsmath,amssymb,amsfonts,amsthm}
\usepackage{algorithmic}
\usepackage{pgfplots}
\usepackage{graphicx}
\usepackage{textcomp}
\usepackage{xcolor}
\usepackage{txfonts}
\usepackage{listings}
\usepackage{enumitem}
\usepackage{mathtools}
\usepackage{gensymb}
\usepackage{comment}
\usepackage[breaklinks=true]{hyperref}
\usepackage{tkz-euclide} 
\usepackage{listings}
% \usepackage{gvv}                                        
\def\inputGnumericTable{}                                 
\usepackage[latin1]{inputenc}                                
\usepackage{color}                                            
\usepackage{array}                                            
\usepackage{longtable}                                       
\usepackage{calc}                                             
\usepackage{multirow}                                         
\usepackage{hhline}                                           
\usepackage{ifthen}                                           
\usepackage{lscape}

\renewcommand{\thefigure}{\theenumi}
\renewcommand{\thetable}{\theenumi}
\setlength{\intextsep}{10pt} % Space between text and floats


\numberwithin{equation}{enumi}
\numberwithin{figure}{enumi}
\renewcommand{\thetable}{\theenumi}
\pgfplotsset{compat=1.18}


% Marks the beginning of the document
\begin{document}
\bibliographystyle{IEEEtran}

\title{Assignment 2}
\author{EE24BTECH11049 \\ Patnam Shariq Faraz Muhammed}

% \maketitle
% \newpage
% \bigskip
{\let\newpage\relax\maketitle}

\begin{enumerate}

    %1st Question
    \item 
    A bar is subjected to fluctuating tensile load from $20 kN$ to $100 kN$. The material has yield strength of $240 MPa$ and endurance limit in reversed bending is $160 MPa$. According to the Soderberg principle, the area of cross-section in $mm^2$ of the bar for a factor of safety of 2 is
    \hfill{\brak{\text{2007-ME}}}
    \begin{multicols}{4}
        \begin{enumerate}
            \item $400$
            \item $600$
            \item $750$
            \item $1000$
        \end{enumerate}
    \end{multicols}

    %2nd Question
    \item 
    A simply supported beam of length $L$ is subjected to a varying distributed load $\sin{\frac{3\pi x}{L}} Nm^{-1}$, where the distance $x$ is measured from the left support. The magnitude of the vertical reaction force in $N$ at the left support is
    \hfill{\brak{\text{2007-ME}}}

    \begin{multicols}{4}
        \begin{enumerate}
            \item Zero
            \item $\frac{L}{3\pi}$
            \item $\frac{L}{\pi}$
            \item $\frac{2L}{\pi}$
        \end{enumerate}
    \end{multicols}

    %3rd Question
    \item 
    Two large diffuse gray parallel plates, separated by a small distance, have surface temperatures of $400 K$ and $300 K$. If the emissivities of the surfaces are $0.8$ and the Stefan-Boltzmann constant is $5.67 \times 10^{-8}\frac{W}{m^2K^4}$, the net radiation heat exchange rate in $\frac{kW}{m^2}$ between the two plates is
    \hfill{\brak{\text{2007-ME}}}

    \begin{multicols}{4}
        \begin{enumerate}
            \item $0.66$
            \item $0.79$
            \item $0.99$
            \item $3.96$
        \end{enumerate}
    \end{multicols}

    %4th Question
    \item 
    A hinged gate of length $5 m$, inclined at $30^\degree$ with the horizontal and with water mass on its left, is shown in the figure below. Density of water is $1000 \frac{kg}{m^3}$. The minimum mass of the gate in kg per unit width \brak{\text{perpendicular to the plane of paper}}, required to keep it closed is

    \begin{figure}[H]
    \centering
    \resizebox{0.7\textwidth}{!}{\begin{circuitikz}
    \tikzstyle{every node}=[font=\large]
    \draw (1.75,6.75) to[short] (11,6.75);
    \draw [line width=1.6pt, short] (7.25,9.75) -- (11,6.75);
    \draw (6,10.25) to[short] (7.75,10.25);
    \draw (6.5,10.25) to[short] (6.5,9.75);
    \draw (7.25,10.25) to[short] (7.25,9.75);
    \draw [short] (6.5,9.75) -- (6.75,9.5);
    \draw [short] (7.25,9.75) -- (7,9.5);
    \draw [short] (6.75,9.5) -- (7,9.5);
    \draw (6.5,9.75) to[short] (1.75,9.75);
    \draw (1.75,9.75) to[short] (1.75,6.75);
    \draw  (6.875,9.9) circle (0.25cm);
    \draw (7,10) to[short] (8.5,11.5);
    \draw (11,6.75) to[short] (12.5,8.25);
    \draw [<->, >=Stealth] (8.5,11.25) -- (12.25,8.25);
    \draw [line width=1.5pt, short] (6,10.25) -- (6.25,10.5);
    \draw [line width=1.5pt, short] (6.25,10.25) -- (6.5,10.5);
    \draw [line width=1.5pt, short] (6.5,10.25) -- (6.75,10.5);
    \draw [line width=1.5pt, short] (6.75,10.25) -- (7,10.5);
    \draw [line width=1.5pt, short] (7,10.25) -- (7.25,10.5);
    \draw [line width=1.5pt, short] (7.25,10.25) -- (7.5,10.5);
    \draw [line width=1.5pt, short] (7.5,10.25) -- (7.75,10.5);
    \draw [dashed] (1.75,10) -- (6.5,10);
    \node [font=\large, rotate around={-45:(0,0)}] at (10.5,10) {$5 m$};
    \draw [line width=2pt, short] (7.25,9.75) -- (11,6.75);
    \draw [line width=2pt, short] (7.25,9.75) -- (11,6.75);
    \draw[domain=1.75:6.75,samples=100,smooth, line width=0.2pt] plot (\x,{0.1*sin(10*\x r -1.75 r ) +9.5});
    \draw[domain=1.75:7.75,samples=100,smooth, line width=0.2pt] plot (\x,{0.1*sin(10*\x r -1.75 r ) +9.25});
    \draw[domain=7:7.5,samples=100,smooth, line width=0.2pt] plot (\x,{0.1*sin(10*\x r -7 r ) +9.5});
    \draw[domain=1.75:8,samples=100,smooth, line width=0.2pt] plot (\x,{0.1*sin(10*\x r -1.75 r ) +9});
    \draw[domain=1.75:8.5,samples=100,smooth, line width=0.2pt] plot (\x,{0.1*sin(10*\x r -1.75 r ) +8.75});
    \draw[domain=1.75:8.75,samples=100,smooth, line width=0.2pt] plot (\x,{0.1*sin(10*\x r -1.75 r ) +8.5});
    \draw[domain=1.75:9,samples=100,smooth, line width=0.2pt] plot (\x,{0.1*sin(10*\x r -1.75 r ) +8.25});
    \draw[domain=1.75:9.25,samples=100,smooth, line width=0.2pt] plot (\x,{0.1*sin(10*\x r -1.75 r ) +8});
    \draw[domain=1.75:9.5,samples=100,smooth, line width=0.2pt] plot (\x,{0.1*sin(10*\x r -1.75 r ) +7.75});
    \draw[domain=1.75:10,samples=100,smooth, line width=0.2pt] plot (\x,{0.1*sin(10*\x r -1.75 r ) +7.5});
    \draw[domain=1.75:10.25,samples=100,smooth, line width=0.2pt] plot (\x,{0.1*sin(10*\x r -1.75 r ) +7.25});
    \draw[domain=1.75:10.5,samples=100,smooth, line width=0.2pt] plot (\x,{0.1*sin(10*\x r -1.75 r ) +7});
\end{circuitikz}
}
    \end{figure}
    \hfill{\brak{\text{2007-ME}}}

    \begin{multicols}{4}
        \begin{enumerate}
            \item $5000$
            \item $6600$
            \item $7546$
            \item $9623$
        \end{enumerate}
    \end{multicols}

    %5th Question 
    \item 
    The pressure, temperature and velocity of air flowing in a pipe are $5 bar$, $500 K$ and $50 \frac{m}{s}$ respectively. The specific heats of air at constant pressure and at constant volume are $1.005 \frac{KJ}{kgK}$ and $0.718 \frac{KJ}{kgK}$ respectively. Neglect potential energy. If the pressure and temperature of the surroundings are $1 bar$ and $300 K$, respectively, the available energy in $\frac{kJ}{kg}$ of the air stream is
    \hfill{\brak{\text{2007-ME}}}

    \begin{multicols}{4}
        \begin{enumerate}
            \item $170$
            \item $187$
            \item $191$
            \item $213$
        \end{enumerate}
    \end{multicols}

    %6th Question
    \item
    The probability that a student knows the correct answer to a multiple choice question is $\frac{2}{3}$ . If the student does not know the answer, then the student guesses the answer. The probability of the guessed answer being correct is $\frac{1}{4}$. Given that the student has answered the question correctly, the conditional probability that the student knows the correct answer is
    \hfill{\brak{\text{2007-ME}}}

    \begin{multicols}{4}
        \begin{enumerate}
            \item $\frac{2}{3}$
            \item $\frac{3}{4}$
            \item $\frac{5}{6}$
            \item $\frac{8}{9}$
        \end{enumerate}
    \end{multicols}

    %7th Question 
    \item 
    The solution of the differential equation 
    \begin{align*}
        \frac{d^2u}{dx^2} - k\frac{du}{dx} = 0
    \end{align*}
    where $k$ is a constant, subjected to the boundary conditions $u\brak{0} = 0$ and $u\brak{L} = U$, is 
    \hfill{\brak{\text{2007-ME}}}

    \begin{multicols}{2}
        \begin{enumerate}
            \item $u = \displaystyle U \frac{x}{L}$
            \item $u = \displaystyle U\brak{\frac{1-e^{kx}}{1-e^{kL}}}$
            \item $u = \displaystyle U\brak{\frac{1-e^{-kx}}{1-e^{-kL}}}$
            \item $u = \displaystyle U\brak{\frac{1+e^{kx}}{1+e^{kL}}}$
        \end{enumerate}
    \end{multicols}

    %8th Question 
    \item 
    The value of the definite integral
    \begin{align*}
        \int_1^e \sqrt{x}\ln{\brak{x}}\, dx
    \end{align*}
    is
    \hfill{\brak{\text{2007-ME}}}

    \begin{multicols}{4}
        \begin{enumerate}
            \item $\frac{4}{9}\sqrt{e^3} + \frac{2}{9}$
            \item $\frac{2}{9}\sqrt{e^3} - \frac{4}{9}$
            \item $\frac{2}{9}\sqrt{e^3} + \frac{4}{9}$
            \item $\frac{4}{9}\sqrt{e^3} - \frac{2}{9}$
        \end{enumerate}
    \end{multicols}

    \section{Common Data Questions}

    \subsection{Common Data for Questions 48 \& 49: }\ref{48} \ref{49}
   
    A single riveted lap joint of two similar plates as shown in the figure below has the following geometrical
    and material details.
    
    \begin{figure}[H]
    \centering
    \resizebox{0.8\textwidth}{!}{\begin{circuitikz}
    \tikzstyle{every node}=[font=\large]
    \draw (2.5,10.25) to[short] (10,10.25);
    \draw (2.5,7.5) to[short] (10,7.5);
    \draw [<->, >=Stealth] (3.75,10.25) -- (3.75,7.5)node[pos=0.5, fill=white]{W};
    \draw [<->, >=Stealth] (8.5,10.25) -- (8.5,7.5)node[pos=0.5, fill=white]{W};
    \draw [dashed] (5.5,10.25) -- (5.5,7.5);
    \draw [short] (6.5,10.25) -- (6.5,7.5);
    \draw [fill, black] (6,9.75) circle (0.25cm);
    \draw [fill, black] (6,8.875) circle (0.25cm);
    \draw [fill, black] (6,8) circle (0.25cm);
    \draw [short] (2.5,10.25) .. controls (1.75,9.5) and (2,9.75) .. (2.5,9);
    \draw [short] (2.5,9) .. controls (3,8.25) and (3,8) .. (2.5,7.5);
    \draw [short] (10,10.25) .. controls (10.75,9.5) and (10.5,9.5) .. (10,9);
    \draw [short] (10,9) .. controls (9.25,8.25) and (9.5,8) .. (10,7.5);
    \draw (2.5,6.5) to[short] (6.25,6.5);
    \draw (5.75,5.5) to[short] (10,5.5);
    \draw [short] (5.5,6) -- (5.75,5.5);
    \draw [short] (6.25,6.5) -- (6.5,6);
    \draw (5.75,6.5) to[short] (5.75,5.5);
    \draw (6.25,6.5) to[short] (6.25,5.5);
    \draw (6.25,6) to[short] (10,6);
    \draw (2.5,6) to[short] (5.75,6);
    \draw (10,6) to[short] (10,5.5);
    \draw (2.5,6.5) to[short] (2.5,6);
    \draw (1.75,6.5) to[short] (2.25,6.5);
    \draw (1.75,6) to[short] (2.25,6);
    \draw (10.25,6) to[short] (10.75,6);
    \draw (10.25,5.5) to[short] (10.75,5.5);
    \draw [dashed] (6,6.75) -- (6,5.25);
    \draw [->, >=Stealth] (2,7) -- (2,6.5);
    \draw [->, >=Stealth] (2,5.5) -- (2,6);
    \draw [->, >=Stealth] (10.5,6.5) -- (10.5,6);
    \draw [->, >=Stealth] (10.5,5) -- (10.5,5.5);
    \draw [line width=1.6pt, short] (2.5,6.25) -- (2.75,6.5);
    \draw [line width=1.6pt, short] (2.5,6) -- (3,6.5);
    \draw [line width=1.6pt, short] (2.75,6) -- (3.25,6.5);
    \draw [line width=1.6pt, short] (3,6) -- (3.5,6.5);
    \draw [line width=1.6pt, short] (3.25,6) -- (3.75,6.5);
    \draw [line width=1.6pt, short] (3.5,6) -- (4,6.5);
    \draw [line width=1.6pt, short] (3.75,6) -- (4.25,6.5);
    \draw [line width=1.6pt, short] (4,6) -- (4.5,6.5);
    \draw [line width=1.6pt, short] (4.5,6) -- (5,6.5);
    \draw [line width=1.6pt, short] (4.25,6) -- (4.75,6.5);
    \draw [line width=1.6pt, short] (4.75,6) -- (5.25,6.5);
    \draw [line width=1.6pt, short] (5,6) -- (5.5,6.5);
    \draw [line width=1.6pt, short] (5.25,6) -- (5.75,6.5);
    \draw [line width=1.6pt, short] (5.5,6) -- (5.75,6.25);
    
    \draw [ line width=1.6pt](6.25,6) to[short] (6.75,5.5);
    \draw [ line width=1.6pt](6.5,6) to[short] (7,5.5);
    \draw [ line width=1.6pt](7,6) to[short] (7.5,5.5);
    \draw [ line width=1.6pt](6.75,6) to[short] (7.25,5.5);
    \draw [ line width=1.6pt](7.25,6) to[short] (7.75,5.5);
    \draw [ line width=1.6pt](7.5,6) to[short] (8,5.5);
    \draw [ line width=1.6pt](7.75,6) to[short] (8.25,5.5);
    \draw [ line width=1.6pt](8,6) to[short] (8.5,5.5);
    \draw [ line width=1.6pt](8.25,6) to[short] (8.75,5.5);
    \draw [ line width=1.6pt](8.5,6) to[short] (9,5.5);
    \draw [ line width=1.6pt](8.75,6) to[short] (9.25,5.5);
    \draw [ line width=1.6pt](9,6) to[short] (9.5,5.5);
    \draw [ line width=1.6pt](9.25,6) to[short] (9.75,5.5);
    \draw [ line width=1.6pt](9.5,6) to[short] (10,5.5);
    \draw [ line width=1.6pt](9.75,6) to[short] (10,5.75);
    \draw [ line width=1.6pt](6.25,5.75) to[short] (6.5,5.5);
    \node [font=\large] at (11.5,9) {P};
    \node [font=\large] at (1,9) {P};
    \node [font=\large] at (2,6.25) {$t$};
    \node [font=\large] at (10.5,5.75) {$t$};
    \draw [->, >=Stealth] (10.25,9) -- (11,9);
    \draw [->, >=Stealth] (2.25,9) -- (1.5,9);
\end{circuitikz}
}
    \end{figure}

    width of the plate $w = 200 mm$, thickness of the plate $t = 5 mm$, number of rivets $n = 3$, diameter of the rivet $d_r = 10 mm$, diameter of the rivet hole $d_h = 11 mm$, allowable tensile stress of the plate $\sigma_p = 200 MPa$, allowable shear stress of the rivet $\sigma_s = 100 MPa$ and allowable bearing stress of the rivet $\sigma_c = 150 MPa$.\\

    %9th Question
    \item 
    If the rivets are to be designed to avoid crushing failure, the maximum permissible load $P$ in $kN$ is \label{48}
    \hfill{\brak{\text{2007-ME}}}

    \begin{multicols}{4}
        \begin{enumerate}
            \item $7.50$
            \item $15.00$
            \item $22.50$
            \item $30.00$
        \end{enumerate}
    \end{multicols}

    %10th Question
    \item 
    If the plates are to be designed to avoid tearing failure, the maximum permissible load $P$ in $kN$ is \label{49}
    \hfill{\brak{\text{2007-ME}}}

    \begin{multicols}{4}
        \begin{enumerate}
            \item $83$
            \item $125$
            \item $167$
            \item $501$
        \end{enumerate}
    \end{multicols}

    \subsection{Common Data for Questions 50 \& 51: }\ref{50} \ref{51}

    Water $\brak{\text{specific heat, } c_p = 4.18 \frac{KJ}{kgK}}$ nters a pipe at a rate of $0.01 \frac{kg}{s}$ and a temperature of $20^\degree C$. The pipe, of diameter $50 mm$ and length $3 m$, is subjected to a wall heat flux $q^{\prime\prime}$ in $\frac{W}{m^2}$\\
    
    %11th Question
    \item 
    If$q_W^{\prime\prime} = 2500x $, where $x$ is in m and in the direction of flow $\brak{x = 0\text{ at the inlet}}$, the bulk mean temperature of the water leaving the pipe in $^\degree C$ is \label{50}
    \hfill{\brak{\text{2007-ME}}}

    \begin{multicols}{4}
        \begin{enumerate}
            \item $42$
            \item $62$
            \item $74$
            \item $104$
        \end{enumerate}
    \end{multicols}

    %12th Question
    \item
    If $q_W^{\prime\prime} = 5000$ and the convection heat transfer coefficient at the pipe outlet is $1000\frac{W}{m^2K}$, the temperature in $^\degree C$ at the inner surface of the pipe at the outlet is \label{51}
    \hfill{\brak{\text{2007-ME}}}

    \begin{multicols}{4}
        \begin{enumerate}
            \item $71$
            \item $76$
            \item $79$
            \item $81$
        \end{enumerate}
    \end{multicols}

    \section{Linked Answer Questions}
    \subsection{Statement for Linked Answer Questions 52 \& 53: }\ref{52}
        
    In orthogonal turning of a bar of $100 mm$ diameter with a feed of $0.25 \frac{mm}{rev}$, depth of cut of $4 mm$ and cutting velocity of $90 \frac{m}{min}$, it is observed that the main \brak{\text{tangential}} cutting force is perpendicular to the friction force acting at the chip-tool interface. The main \brak{\text{tangential}} cutting force is $1500 N$.\\

    %13th Question
    \item 
    The orthogonal rake angle of the cutting tool in $degree$ is \label{52}
    \hfill{\brak{\text{2007-ME}}}
    \begin{multicols}{4}
        \begin{enumerate}
            \item zero
            \item $3.58$
            \item $5$
            \item $7.16$
        \end{enumerate}
    \end{multicols}

\end{enumerate}
\end{document}
