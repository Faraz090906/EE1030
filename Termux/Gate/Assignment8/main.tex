\let\negmedspace\undefined
\let\negthickspace\undefined
\documentclass[journal]{IEEEtran}
\usepackage[a5paper, margin=10mm, onecolumn]{geometry}
%\usepackage{lmodern} % Ensure lmodern is loaded for pdflatex
\usepackage{tfrupee} % Include tfrupee package

\setlength{\headheight}{1cm} % Set the height of the header box
\setlength{\headsep}{0mm}     % Set the distance between the header box and the top of the text
\usepackage{xparse}
\usepackage{gvv-book}
\usepackage{gvv}
\usepackage{cite}
\usepackage{amsmath,amssymb,amsfonts,amsthm}
\usepackage{algorithmic}
\usepackage{graphicx}
\usepackage{textcomp}
\usepackage{xcolor}
\usepackage{txfonts}
\usepackage{listings}
\usepackage{enumitem}
\usepackage{mathtools}
\usepackage{gensymb}
\usepackage{comment}
\usepackage[breaklinks=true]{hyperref}
\usepackage{tkz-euclide} 
\usepackage{listings}
% \usepackage{gvv}                                        
\def\inputGnumericTable{} 
\usepackage[latin1]{inputenc}                                
\usepackage{color}                                            
\usepackage{array}                                            
\usepackage{longtable}                                       
\usepackage{calc}                                             
\usepackage{multirow}                                         
\usepackage{hhline}                                           
\usepackage{ifthen}                                           
\usepackage{lscape}

\begin{document}

\bibliographystyle{IEEEtran}
\vspace{3cm}

\title{Assignment 7}
\author{EE24BTECH11049 \\ Patnam Shariq Faraz Muhammed}
% \maketitle
% \newpage
% \bigskip
{\let\newpage\relax\maketitle}
\begin{enumerate}
	\item If $\to$ denotes increasing order of intensity, then the meaning of the words $\sbrak{\text{dry} \to \text{arid} \to \text{parched}}$ is analogous to $\sbrak{\text{diet} \to \text{fast} \to \rule{1cm}{0.1pt}}$. Which one of the given options is approriate to fill the blank ?
		\begin{enumerate}
				\begin{multicols}{4}
				\item starve
				\item reject
				\item feast
				\item deny
				\end{multicols}
		\end{enumerate}
	\item If two distinct non-zero variables $x$ and $y$ are such that $\brak{x+y}$ is proportional to $\brak{x-y}$ then the value of $\frac{x}{y}$ 
		\begin{enumerate}
			\item depends on $xy$
			\item depends only on $x$ and not on $y$
			\item depends only on $y$ and not on $x$
			\item is a constant
		\end{enumerate}
	\item Consider the following sample of samples : \\
		9, 18, 11, 14, 15, 17, 10, 69, 11, 13 \\
		The median of the sample is
		\begin{enumerate}
				\begin{multicols}{4}
				\item 13.5
				\item 14
				\item 11
				\item 18.7
				\end{multicols}
		\end{enumerate}
	\item The number of coins of rupees 1, rupees 5 and rupees 10 denominations that a person has are in the ratio 5:3:13. Of the total amount, the percentage of money in 5 rupee coins is
		\begin{enumerate}
				\begin{multicols}{4}
				\item 21 \%
				\item $ 14 \frac{2}{7} \%$
				\item 10 \%
				\item 30 \%
				\end{multicols}
		\end{enumerate}
	\item For positive non-zero real variables $p$ and $q$, if 
		$$ \log{p^2 + q^2} = \log{p} + \log{q} + 2 \log{3} $$
		then, the value of $\frac{p^4 + q^4}{p^2 q^2}$ is
		\begin{enumerate}
				\begin{multicols}{4}
				\item 79
				\item 81
				\item 9
				\item 83
				\end{multicols}
		\end{enumerate}
	\item In the given text, the blanks are numbered (i) - (iv). Select the best match for all the blanks. \\
		Steve was advised to keep his head (i) \rule{1cm}{0.1pt} before heading (ii) \rule{1cm}{0.1pt} to bat ; for, while he had a head (iii) \rule{1cm}{0.1pt} batting, he could only do so with a cool head (iv) \rule{1cm}{0.1pt} his shoulders.
		\begin{enumerate}
			\item (i) down (ii) down (iii) on (iv) for
			\item (i) on (ii) down (iii) for (iv) on
			\item (i) down (ii) out (iii) for (iv) on
			\item (i) on (ii) out (iii) on (iv) for
		\end{enumerate}
	\item A rectangular paper sheet of dimensions $ 54 cm \times 4 cm$ is taken. The two longer edges of the sheet are joined together to create a cylindrical tube. A cube whose surface area is eqal to the area of the sheet is also taken. \\
		Then, the ratio of the volume of the cylindrical tube to the volume of the cube is
		\begin{enumerate}
				\begin{multicols}{4}
				\item $\frac{1}{\pi}$
				\item $\frac{2}{\pi}$
				\item $\frac{3}{\pi}$
				\item $\frac{4}{\pi}$
				\end{multicols}
		\end{enumerate}
	\item The pie chart presents the percentage contribution of different macro-nutrients to a typical 200 kCal diet of a person. \\
		\begin{figure}[H]
			\centering
			\begin{circuitikz}
\tikzstyle{every node}=[font=\normalsize]
\draw  (5.75,11) circle (3.5cm);
\draw [short] (5.75,10.75) -- (5.75,14.5);
\draw [short] (5.75,10.75) -- (3.75,8.25);
\draw [short] (5.75,10.75) -- (8.5,8.75);
\draw [short] (5.75,10.75) -- (2.25,11);
\draw [short] (5.75,10.75) -- (4.25,14);
\node [font=\normalsize] at (7.5,11.25) {Carbohydrates 35 \%};
\node [font=\normalsize] at (5.25,13.5) {Trans-fat 5 \%};
\node [font=\normalsize] at (3.75,11.75) {Saturated fat 20 \%};
\node [font=\normalsize] at (4,10.5) {Unsaturated fat 20 \%};
\node [font=\normalsize] at (6,8.75) {Proteins 20 \%};
\end{circuitikz}

		\end{figure}
		The typical energy density ( kCal/g ) of these macro-nutrients is given in the table. \\
		\begin{table}[H]
\centering
\begin{tabular}{|c|c|}
\hline
\textbf{Macronutrient} & \textbf{Energy density ( kCal/g )} \\
\hline
Carbohydrates & 4 \\
\hline
Proteins & 4 \\
\hline
Unsaturated fat & 9 \\
\hline
Saturated fat & 9 \\
\hline
Trans fat & 9 \\
\hline
\end{tabular}
\end{table}

		The total fat ( all three types ), in grams, this person consumes is
		\begin{enumerate}
				\begin{multicols}{4}
				\item 44.4
				\item 77.8
				\item 100
				\item 3,600
				\end{multicols}
		\end{enumerate}
	\item A rectangular paper of $20 cm \times 8 cm$ is folded 3 times. Each fold is made along the line of symmetry, which is perpendicular to its long edge. The perimeter of the final folded sheet ( in cm ) is
		\begin{enumerate}
				\begin{multicols}{4}
				\item 18
				\item 24
				\item 20
				\item 21
				\end{multicols}
		\end{enumerate}
	\item The least number of squares to be added in the figure to make AB a line of symmetry is 
		\begin{figure}[H]
			\centering
			\begin{circuitikz}
\tikzstyle{every node}=[font=\normalsize]
\draw  (4,11.5) rectangle (4.5,12);
\draw  (4,11.5) rectangle (3.5,11);
\draw  (4,11) rectangle (4.5,10.5);
\draw  (5.25,11.5) rectangle (5.75,11);
\draw  (5.75,11) rectangle (6.25,11.5);
\draw  (5.75,11) rectangle (6.25,10.5);
\node [font=\normalsize] at (2.5,12) {A};
\node [font=\normalsize] at (7,12) {B};
\draw [dashed] (2,11.5) -- (7.5,11.5);
\end{circuitikz}

		\end{figure}
		\begin{enumerate}
				\begin{multicols}{4}
				\item 6
				\item 4
				\item 5
				\item 7
				\end{multicols}
		\end{enumerate}
	\item The following system of linear equations 
		$$ 7x - 3y + z = 0 $$
		$$ 3x - y + z = 0 $$
		$$ x - y - z = 0 $$
		has
		\begin{enumerate}
				\begin{multicols}{2}
				\item infinitely many solutions
				\item a unique solution
				\item no solution
				\item three solutions
				\end{multicols}
		\end{enumerate}
	\item The acceleration of a body travelling in a straight line is given by 
		$$ \alpha = -C_1 - C_2 v^2 $$
		where $v$ is the velocity, and $C_1, C_2$ are positive constants. Starting with an initial positive velocity $v_0$, the distance travelled by the body before before coming to rest for the first time :
		\begin{enumerate}
				\begin{multicols}{2}
				\item $\frac{1}{2C_2} ln \brak{1 + \frac{C_2}{C_1} v_{0}^2 }$
				\item $\frac{1}{2C_2} ln \brak{1 - \frac{C_2}{C_1} v_{0}^2 }$
				\item $\frac{1}{2C_2} ln \brak{C_1 + C_2 v_{0}^2 }$
				\item $\frac{1}{2C_2} ln \brak{1 + C_2 v_{0}^2 }$
				\end{multicols}
		\end{enumerate}
	\item The three dimensional stress-strain relationship for an isotropic material is given as 
		$$\myvec{
\sigma_{xx} \\
\sigma_{yy} \\
\sigma_{zz} \\
\tau_{yz} \\
\tau_{xz} \\
\tau_{xy}
}
=
\myvec{
P & Q & Q & 0 & 0 & 0 \\
Q & P & Q & 0 & 0 & 0 \\
Q & Q & P & 0 & 0 & 0 \\
0 & 0 & 0 & R & 0 & 0 \\
0 & 0 & 0 & 0 & R & 0 \\
0 & 0 & 0 & 0 & 0 & R
}
\myvec{
\epsilon_{xx} \\
\epsilon_{yy} \\
\epsilon_{zz} \\
\epsilon_{yz} \\
\epsilon_{xz} \\
\epsilon_{xy}
}$$
		where, P, Q and R are the three elastic constants,$\sigma \text{and} \tau$ represent normal and shear stresses, and $\epsilon \text{and} \gamma$ represent normal and engineering shear strains. Which one of the following options is correct ?
		\begin{enumerate}
				\begin{multicols}{4}
				\item $R = \frac{P - Q}{2}$
				\item $R = \frac{Q - P}{2}$
				\item $Q = \frac{P - R}{2}$
				\item $Q = \frac{R - P}{2}$
				\end{multicols}
		\end{enumerate}

\end{enumerate}
\end{document}
u
