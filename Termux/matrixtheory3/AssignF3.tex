\let\negmedspace\undefined
\let\negthickspace\undefined
\documentclass[journal]{IEEEtran}
\usepackage[a5paper, margin=10mm, onecolumn]{geometry}
%\usepackage{lmodern} % Ensure lmodern is loaded for pdflatex
\usepackage{tfrupee} % Include tfrupee package

\setlength{\headheight}{1cm} % Set the height of the header box
\setlength{\headsep}{0mm}     % Set the distance between the header box and the top of the text

\usepackage{gvv-book}
\usepackage{gvv}
\usepackage{cite}
\usepackage{amsmath,amssymb,amsfonts,amsthm}
\usepackage{algorithmic}
\usepackage{graphicx}
\usepackage{textcomp}
\usepackage{xcolor}
\usepackage{txfonts}
\usepackage{listings}
\usepackage{enumitem}
\usepackage{mathtools}
\usepackage{gensymb}
\usepackage{comment}
\usepackage[breaklinks=true]{hyperref}
\usepackage{tkz-euclide} 
\usepackage{listings}
% \usepackage{gvv}                                        
\def\inputGnumericTable{}                                 
\usepackage[latin1]{inputenc}                                
\usepackage{color}                                            
\usepackage{array}                                            
\usepackage{longtable}                                       
\usepackage{calc}                                             
\usepackage{multirow}                                         
\usepackage{hhline}                                           
\usepackage{ifthen}                                           
\usepackage{lscape}

\renewcommand{\thefigure}{\theenumi}
\renewcommand{\thetable}{\theenumi}
\setlength{\intextsep}{10pt} % Space between text and floats


\numberwithin{equation}{enumi}
\numberwithin{figure}{enumi}
\renewcommand{\thetable}{\theenumi}

% Marks the beginning of the document
\begin{document}
\bibliographystyle{IEEEtran}

\title{Assignment 2 \\ Chapter-15: \\ Matrices and Determinants}
\author{EE24BTECH11049 \\ Patnam Shariq Faraz Muhammed}

% \maketitle
% \newpage
% \bigskip
{\let\newpage\relax\maketitle}

\begin{enumerate}

\item
%1st question
	If $a>0$ and discriminant of $ax^{2}+2bx+c$ is -ve, then
	$\mydet{a & b & ax+b \\ b & c & bx+c \\ ax+b & bx+c & 0}$ is equal to 
	
	\hfill[2002]	

	\begin{enumerate}
		\item +ve
		\item $\brak{ac-b^2}\brak{ax^2+2bx+c}$
		\item -ve
		\item $0$
	\end{enumerate}

\item
%2nd question
	If the system of linear equations $x+2ay+az = 0$; $x+3by+bz = 0$; $x+4cy+cz = 0$; has a non-zero solution, then a,b,c.

	\hfill[2003]

	\begin{enumerate}
                \item satisfy $a+ 2b+3c = 0$
                \item are in A.P
                \item are in G.P
                \item are in H.P
        \end{enumerate} 	

\item                    
%3rd question
	If 1, $\omega, \omega^{2}$ are the cube roots of unity, then
	$\Delta = \mydet{1 & \omega^{n} & \omega^{2n} \\ \omega^{n} & \omega^{2n} & 1 \\ \omega^{2n} & 1 & \omega^{n}}$ is equal to 
        
        \hfill[2003]
        
        \begin{enumerate}
		\item $\omega^{2}$ 
                \item 0                      
                \item 1
                \item $\omega$
        \end{enumerate}


\item
%4th question
        If $A=\myvec{a & b \\ b & a}$ and $A^{2}=\myvec{\alpha & \beta \\ \beta & \alpha}$, then 
        
        \hfill[2003]
                                             
        \begin{enumerate}
		\item $\alpha=2ab, \beta=a^{2}+b^{2}$
                \item $\alpha=a^{2}+b^{2}, \beta=ab$
                \item $\alpha=a^{2}+b^{2}, \beta=2ab$
                \item $\alpha=a^{2}+b^{2}, \beta=a^{2}-b^{2}$
        \end{enumerate}

		
\end{enumerate}


\end{document}
